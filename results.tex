\section{Results}

In this section, we compare the results from Moltres for each benchmark step to
the results reported in the the benchmark paper \cite{tiberga_results_2020}. The
benchmark paper contains results from four different \gls{MSR} simulation
software packages. The authors performed code-to-code verification by reporting
the average discrepancy $\epsilon$ from all software packages for each benchmark
step. The discrepancy $\epsilon_c$ from each software package for each measured
quantity $Q_c$ calculated using the following equation:
%
\begin{align}
    \epsilon_c &= \sqrt{\frac{\sum^{N_p}_{i=1}\left(Q_c(\vec{r_i}) - Q_{avg}
    (\vec{r_i})\right)^2}{\sum^{N_p}_{i=1} Q^2_{avg}(\vec{r_i})}}
    \intertext{where}
    N_p &= 201 \nonumber \\
    &= \mbox{number of sampling points of quantity $Q$,}
    \nonumber \\
    N_c &= \mbox{number of \gls{MSR} software packages,} \nonumber \\
    Q_{avg}(\vec{r_i}) &= \frac{1}{N_c} \sum^{N_c}_{c=1} Q_c(\vec{r_i})
    \nonumber \\
    &= \mbox{average value of $Q$ at $\vec{r_i}$ from all packages.} \nonumber
\end{align}
%
The average discrepancy from all software packages is then $\epsilon =
\frac{1}{N_c} \sum^{N_c}_{c=1} \epsilon_c$.

\subsection{Step 0.1: Velocity field}

Table \ref{table:vel} shows the discrepancies in the velocity components
$(u_x, u_y)$ from Moltres and the corresponding average discrepancies from the
benchmark. Moltres performed excellently as all four discrepancy values from
Moltres fall below the average discrepancy from the benchmark.

\begin{table}[htbp!]
	\caption{Discrepancies in $(u_x, u_y)$ from Step 0.1.}
	\centering
	\small
	\setlength\tabcolsep{1.5pt}
	\begin{tabular}{c c S S}
		\toprule
		\multirow{2}{*}{\textbf{Observable}} & \multirow{2}{*}{\textbf{Code}} & \multicolumn{2}{c}{\textbf{Discrepancies [\%]}} \\
		& & {Along AA'} & {Along BB'} \\
		\midrule
		\multirow{2}{*}{$u_x$ [m s$^{-1}$]} & Moltres & 0.247 & 0.266 \\
		& Benchmark & 0.253 & 0.318 \\
        \midrule
		\multirow{2}{*}{$u_y$ [m s$^{-1}$]} & Moltres & 0.540 & 0.467 \\
		& Benchmark & 0.598 & 0.795 \\
		\bottomrule
	\end{tabular}
	\label{table:vel}
\end{table}

\subsection{Step 0.2: Neutronics}

Table \ref{table:nts} shows the discrepancies in the fission rate density
$\sum^G_i \Sigma_{f,i} \phi_i(\vec{r})$ from Moltres and the corresponding
average discrepancies from the benchmark.
Moltres reports a negligibly worse discrepancy in the fission rate density.
We also observe in table \ref{table:rho02} that the $\rho$ value from Moltres
falls within the range of $\rho$ values from the benchmark. We attribute some
of the overall discrepancy to the zeroth-order shape functions used to
approximate the precursor concentrations. Zeroth-order shape functions
perform worse in regions with large gradients observed near the
boundaries of the homogeneous domain with no reflectors.

\begin{table}[htbp!]
	\caption{Discrepancies in the fission rate density from Step 0.2.}
	\centering
	\small
	\setlength\tabcolsep{1.5pt}
	\begin{tabular}{c c S}
		\toprule
		\multirow{2}{*}{\textbf{Observable}} & \multirow{2}{*}{\textbf{Code}} & \multicolumn{1}{c}{\textbf{Discrepancies [\%]}} \\
		& & {Along AA'} \\
		\midrule
		\multirow{2}{*}{$\sum^G_i \Sigma_{f,i} \phi_i(\vec{r})$
		[m$^{-3}\cdot$s$^{-1}$]} & Moltres & 0.313 \\
		& Benchmark & 0.285\\
		\bottomrule
	\end{tabular}
	\label{table:nts}
\end{table}

\begin{table}[htbp!]
    \caption{Reactivity values from Step 0.2.}
    \centering
    \footnotesize
    \setlength\tabcolsep{1.5pt}
    \begin{tabular}{l S}
        \toprule
        \textbf{Code} & {$\rho$} \\
        \midrule
        Moltres \hspace{3cm} & 465.6 \\
        CNRS-$SP_1$ & 411.3 \\
        CNRS-$SP_3$ & 353.7 \\
        Polimi & 421.2 \\
        PSI & 411.7 \\
        TUD-$S_2$ & 482.6 \\
        TUD-$S_6$ & 578.1 \\
        \bottomrule
    \end{tabular}
    \label{table:rho02}
\end{table}

\subsection{Step 0.3: Temperature}

Table \ref{table:temp} shows the discrepancies in the temperature distribution
from Moltres and the corresponding average discrepancies from the benchmark.
Moltres reported discrepancy values that closely follow the benchmark average
discrepancies. The temperature values at the end of BB', near the top boundary,
deviated significantly (912 K from Moltres
vs ~935 K from the benchmark) because the finite element method struggles with
the velocity boundary condition discontinuity on the top left corner of the
domain. As such, we observe the aforementioned deviation which is located
directly downstream of the discontinuity.

\begin{table}[h!]
	\caption{Discrepancies in the temperature distribution from Step 0.3.}
	\centering
	\small
	\setlength\tabcolsep{1.5pt}
	\begin{tabular}{c c S S}
		\toprule
		\multirow{2}{*}{\textbf{Observable}} & \multirow{2}{*}{\textbf{Code}} & \multicolumn{2}{c}{\textbf{Discrepancies [\%]}} \\
		& & {Along AA'} & {Along BB'} \\
		\midrule
		\multirow{2}{*}{$T$ [K]} & Moltres & 0.090 & 0.164 \\
		& Benchmark & 0.085 & 0.083 \\
		\bottomrule
	\end{tabular}
	\label{table:temp}
\end{table}

\subsection{Step 1.1: Circulating fuel}

Table \ref{table:temp} shows the discrepancies in the delayed neutron source
from Moltres and the corresponding average discrepancies from the benchmark.
Moltres performs relatively worse than the benchmark average but the
discrepancies are still on the same order of magnitude. We attribute this to
the same zeroth-order shape function approximation for the precursor
concentration that we mentioned in Step 0.2.

On the other hand, we observe in Table \ref{table:rho11} that the change in
$\rho$ relative to Step 0.2 falls within the reported range of $\rho$ values
from the benchmark.

\begin{table}[htbp!]
	\caption{Discrepancies in the delayed neutron source from Step 1.1.}
	\centering
	\small
	\setlength\tabcolsep{1.5pt}
	\begin{tabular}{c c S S}
		\toprule
		\multirow{2}{*}{\textbf{Observable}} & \multirow{2}{*}{\textbf{Code}} & \multicolumn{2}{c}{\textbf{Discrepancies [\%]}} \\
		& & {Along AA'} & {Along BB'} \\
		\midrule
		\multirow{2}{*}{$T$ [K]} & Moltres & 0.606 & 0.378 \\
		& Benchmark & 0.346 & 0.294 \\
		\bottomrule
	\end{tabular}
	\label{table:circ}
\end{table}

\begin{table}[htbp!]
    \caption{Reactivity change in Step 1.1, relative to Step 0.2.}
    \centering
    \footnotesize
    \setlength\tabcolsep{1.5pt}
    \begin{tabular}{l S}
        \toprule
        \textbf{Code} & {$\rho - \rho_{s0.2}$} \\
        \midrule
        Moltres \hspace{3cm} & -62.9 \\
        CNRS-$SP_1$ & -62.5 \\
        CNRS-$SP_3$ & -62.6 \\
        Polimi & -62.0 \\
        PSI & -63.0 \\
        TUD-$S_2$ & -62.0 \\
        TUD-$S_6$ & -60.7 \\
        \bottomrule
    \end{tabular}
    \label{table:rho11}
\end{table}

\subsection{Step 1.2: Power coupling}

Table \ref{table:power} shows the discrepancies in the temperature distribution
and change in fission rate density relative to Step 0.2
from Moltres and the corresponding average discrepancies from the benchmark.
For this step, Moltres performed better than the benchmark average with the
exception of the temperature distribution along BB', the cause of which we
addressed in Step 0.3.

\begin{table}[htbp!]
	\caption{Discrepancies in the temperature
	distribution and change in fission rate density relative to Step 0.2 from
	Step 1.2.}
	\centering
	\small
	\setlength\tabcolsep{1.5pt}
	\begin{tabular}{c c S S}
		\toprule
		\multirow{2}{*}{\textbf{Observable}} & \multirow{2}{*}{\textbf{Code}} & \multicolumn{2}{c}{\textbf{Discrepancies [\%]}} \\
		& & {Along AA'} & {Along BB'} \\
		\midrule
		\multirow{2}{*}{$T$ [K]} & Moltres & 0.076 & 0.179 \\
		& Benchmark & 0.095 & 0.089 \\
        \midrule
		\multirow{2}{*}{$\Delta\left[\sum^G_i \Sigma_{f,i} \phi_i(\vec{r})
		\right]_{s_{1.2}-s_{0.2}}$ [m s$^{-1}$]} & Moltres & 1.11 & 1.08 \\
		& Benchmark & 1.58 & 1.13 \\
		\bottomrule
	\end{tabular}
	\label{table:power}
\end{table}
