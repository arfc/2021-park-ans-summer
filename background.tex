\section{Background \& Methodology}

\subsection{CNRS Benchmark}

The ``CNRS benchmark'' \cite{tiberga_results_2020} is a numerical benchmark for
software dedicated to modeling \glspl{MSR}. It consists of three ``phases'' and
eight ``steps'' in total. Each step is a well-defined subproblem for
systematically assessing the capabilities of \gls{MSR} software and pinpointing
the source of discrepancies between software. Phase 0 consists
single-physics problems in fluid dynamics, neutronics, and temperature,
respectively. Phase 1 consists of coupled, steady-state eigenvalue problems.
Lastly, Phase 2 consists of a coupled, transient scenario.

The domain geometry is a 2m by 2m square cavity filled with LiF-BeF$_2$-UF$_4$
molten salt at an initial temperature of 900K \cite{tiberga_results_2020}.
Standard vacuum boundary conditions apply for neutron flux along the
boundaries, with homogeneous boundary conditions for delayed neutron
precursors. No-slip boundary conditions apply for flow in the cavity, except
the top boundary when the subproblem imposes forced convection via lid-driven
cavity flow. For temperature, all boundaries are insulated and we simulate salt
cooling with the following volumetric heat sink equation:
%
\begin{align}
    q'''(\vec{r}) &= \gamma \left(900 - T(\vec{r})\right),
    \intertext{where}
    q''' &= \mbox{volumetric heat sink [W$\cdot$m$^{-3}$],}
    \nonumber \\
    \gamma &= \mbox{heat transfer coefficient [W$\cdot$m$^{-3}\cdot$K$^{-1}$],}
    \nonumber \\
    T(\vec{r}) &= \mbox{temperature at point $(\vec{r})$ [K].} \nonumber
\end{align}

Tiberga et al. \cite{tiberga_results_2020} used Serpent 2
\cite{leppanen_serpent_2014} with the JEFF-3.1 library
\cite{koning_jeff-31_2006} to generate multigroup neutronics data for the
LiF-BeF$_2$-UF$_4$ salt in the domain at 900K. They condensed the neutronics
data into six energy groups and eight precursor groups. Their paper contains
all relevant data \cite{tiberga_results_2020}. The benchmark prescribes the
following equations to govern the temperature dependence in the cross sections
and the neutron diffusion coefficients:
%
\begin{align}
    \Sigma_i (T) &= \Sigma_i(T_{ref})
    \frac{\rho_{fuel}(T)}{\rho_{fuel}(T_{ref})}, \\
    D (T) &= D(T_{ref})
    \frac{\rho_{fuel}(T_{ref})}{\rho_{fuel}(T)},
    \intertext{where}
    \Sigma_i &= \mbox{relevant macroscopic cross section [cm${-1}$],}
    \nonumber \\
    D &= \mbox{neutron diffusion coefficient [cm$^2\cdot$s$^{-1}$],}   
    \nonumber \\
    \rho_{fuel} &= \mbox{density of the fuel salt [kg$\cdot$m$^{-3}$],}
    \nonumber \\
    T_{ref} &= \mbox{reference temperature} = 900\mbox{ K}. \nonumber
\end{align}

The benchmark also prescribes for incompressible Navier-Stokes and the
Boussinesq approximation for buoyancy when evaluating the salt flow in the
domain, but places no restrictions on the type of neutronics model.
The following subsections briefly detail each benchmark subproblem.

\subsubsection{Step 0.1: Velocity field}

This step assesses the steady-state incompressible flow solution. $U_{lid}$
refers to the velocity along the top boundary and it is nonzero to induce
lid-driven cavity flow.

\textit{Input parameters}
%
\begin{itemize}
    \itemsep0em
    \item $U_{lid} = 0.5$m$\cdot$s$^{-1}$
\end{itemize}

\textit{Observables}
\begin{itemize}
    \itemsep0em
    \item Velocity components $(u_x, u_y)$ along centerlines AA' and BB'
\end{itemize}

\subsubsection{Step 0.2: Neutronics}

This step assesses the neutronics solution in a static, isothermal fuel
configuration. We use the reactor power $P$ to normalize the neutron flux.

\textit{Input parameters}
%
\begin{itemize}
    \itemsep0em
    \item Static fuel
    \item $T = 900$ K
    \item $P = 1$ GW
\end{itemize}

\textit{Observables}
\begin{itemize}
    \itemsep0em
    \item Fission rate density, $\sum^G_i \Sigma_{f,i} \phi_i(\vec{r})$, along
    centerline AA'
    \item Reactivity, $\rho$
\end{itemize}
where $\phi_i$ is the neutron flux in group $i$.

\subsubsection{Step 0.3: Temperature}

This step assesses the thermal-hydraulics solution borne from the velocity
field and fission heat generation from the previous two steps.

\textit{Input parameters}
%
\begin{itemize}
    \itemsep0em
    \item Fixed velocity field from Step 0.1
    \item Fixed fission heat source distribution from Step 0.2
    \item Heat transfer coefficient, $\gamma = 1 \times 10^6$ W$\cdot$m$^{-3}
    \cdot$K$^{-1}$
\end{itemize}

\textit{Observables}
\begin{itemize}
    \itemsep0em
    \item Temperature distribution $T$ along centerlines AA' and BB'
\end{itemize}

\subsubsection{Step 1.1: Circulating fuel}

This step assesses the neutronics solution under a fixed velocity field and
uniform temperature distribution. We allow the precursors to drift under the
provided velocity field.

\textit{Input parameters}
%
\begin{itemize}
    \itemsep0em
    \item Fixed velocity field from Step 0.1
    \item $T = 900$ K
    \item $P = 1$ GW
\end{itemize}

\textit{Observables}
\begin{itemize}
    \itemsep0em
    \item Delayed neutron source, $\sum_j \lambda_j C_j$, along AA' and BB'
    \item Reactivity change from Step 0.2, $\rho - \rho_{s_{0.2}}$
\end{itemize}
where $\lambda_j$ is the decay constant of the precursors in group $j$ and
$C_j$ is the precursor concentration in group $j$.

\subsubsection{Step 1.2: Power coupling}

This step assesses the coupled neutronics and thermal-hydraulics solutions
under a fixed velocity field. The non-uniform temperature distribution induces
a change in the shape of the neutron flux which in turn further skews the
temperature distribution.

\textit{Input parameters}
%
\begin{itemize}
    \itemsep0em
    \item Fixed velocity field from Step 0.1
    \item $P = 1$ GW
    \item $\gamma = 1 \times 10^6$ W$\cdot$m$^{-3}\cdot$K$^{-1}$
\end{itemize}

\textit{Observables}
\begin{itemize}
    \itemsep0em
    \item Temperature distribution along AA' and BB'
    \item Reactivity change from Step 1.1, $\rho - \rho_{s_{1.1}}$
    \item Change in fission rate density along AA' and BB' with respect to the
    solution from Step 0.2, $\sum^G_i \Sigma_{f,i} \phi_i(\vec{r}) -
    \left[\sum^G_i \Sigma_{f,i} \phi_i(\vec{r})\right]_{s_{0.2}}$
\end{itemize}

\subsubsection{Step 1.3: Buoyancy}

This step assesses the full multiphysics solution with buoyancy-driven flow. 

\textit{Input parameters}
%
\begin{itemize}
    \itemsep0em
    \item $P = 1$ GW
    \item $U_{lid} = 0$
    \item $\gamma = 1 \times 10^6$ W$\cdot$m$^{-3}\cdot$K$^{-1}$
\end{itemize}

\textit{Observables}
\begin{itemize}
    \itemsep0em
    \item Velocity components $(u_x, u_y)$ along AA' and BB'
    \item Temperature distribution $T$ along AA' and BB'
    \item Delayed neutron source, $\sum_j \lambda_j C_j$, along AA' and BB'
    \item Reactivity change from Step 0.2, $\rho - \rho_{s_{0.2}}$
\end{itemize}

\subsubsection{Step 1.4: Full coupling}

This step assesses the full multiphysics solution with buoyancy effects
and external momentum-driven flow from the nonzero lid velocity.

\textit{Input parameters}
%
\begin{itemize}
    \itemsep0em
    \item $\gamma = 1 \times 10^6$ W$\cdot$m$^{-3}\cdot$K$^{-1}$
    \item $P = [0, 1]$ GW at $0.2$ GW intervals
    \item $U_{lid} = [0, 0.5]$ m$\cdot$s$^{-1}$ at 0.1m$\cdot$s$^{-1}$
    intervals
\end{itemize}

\textit{Observables}
\begin{itemize}
    \itemsep0em
    \item Reactivity change from Step 0.2, $\rho - \rho_{s_{0.2}}$ for the
    various permutations of $P$ and $U_{lid}$ values
\end{itemize}

\subsubsection{Step 2.1: Forced convection transient}

This step assesses the transient response (gain and phase shift) of the fully
coupled system to an oscillating heat sink. $\gamma$ is uniformly perturbed
according to a sine function with an amplitude equal to 10\% of the initial
amplitude and at the following frequencies: $[0.0125, 0.025, 0.05, 0.1, 0.2,
0.4, 0.8]$ Hz.

\textit{Initial conditions}
%
\begin{itemize}
    \itemsep0em
    \item Steady-state solution from Step 1.4 with
    $U_{lid} = 0.5 $m$\cdot$s$^{-1}$ and $P = 1$ GW
    \item $\gamma = 1 \times 10^6$ W$\cdot$m$^{-3}\cdot$K$^{-1}$
\end{itemize}

\textit{Observables}
\begin{itemize}
    \itemsep0em
    \item Power gain and phase shift at each perturbation frequency.
\end{itemize}
%
The gain is defined as:
%
\begin{align}
    Gain &= \frac{(P_{max} - P_{avg})/P_{avg}}{(\gamma_{max} - \gamma_{avg})/
    \gamma_{avg}}
\end{align}

\subsection{Moltres}

Moltres \cite{lindsay_introduction_2018} is an open-source, \gls{MOOSE}-based
application designed for multiphysics simulations of \glspl{MSR}. The goal of
making Moltres open-source is to promote quality through transparency and
ease of peer review. The source code \cite{lindsay_moltres_2017} is available
on GitHub \cite{lindsay_moltres_2017}. Moltres leverages on \texttt{git} for
version control, and integrated testing to protect existing capabilities while
concurrently supporting continued code development. Moltres depends on the
\gls{MOOSE} finite element framework for its meshing and parallel, nonlinear
NEWTON-based solver capabilities; as well as common physics modules such as the
Navier-Stokes module \cite{peterson_overview_2017} in \gls{MOOSE}. Therefore,
Moltres by default uses tight coupling methods with implicit time-stepping.
Users can also opt to decouple problems depending on their specific needs.

Moltres solves the multigroup neutron diffusion equation for an arbitrary
number of energy and precursor groups. For this study, we coupled Moltres'
neutronics capabilities with \gls{MOOSE}'s incompressible Navier-Stokes
capabilities \cite{peterson_overview_2017} to
model precursor drift and thermal-hydraulics. We performed all simulations
except Step 1.4 using the Preconditioned \gls{JFNK} method with all physics
fully coupled. Step 1.4 required a transient solve on the thermal-hydraulics
portion of the problem to resolve the coupling between lid-driven flow and
buoyancy effects. We then loosely coupled the thermal-hydraulics transient
solve to the neutronics steady-state criticality solve. For future work, we
could explore and determine the appropriate initial conditions or \gls{MOOSE}'s
nonlinear solver settings to circumvent this issue.

For this study, we discretized the problem domain into a 200 by 200 structured
mesh, resulting in uniform mesh elements of dimensions 1cm by 1cm. We
approximated most of the relevant variables, i.e. neutron fluxes, velocity
components, pressure, and temperature, using first-order Lagrange shape
functions. The only exception is the precursor concentration variables, which
we approximated using zeroth-order monomial shape functions and solved using
the Discontinuous Galerkin finite element method to eliminate spurious
numerical oscillations arising from the high Schmidt number flow. As for the
high Reynolds number flow, we stabilized the incompressible Navier-Stokes
equations using the streamline upwind Petrov-Galerkin and pressure-stabilizing
Petrov-Galerkin stabilization methods \cite{peterson_overview_2017} provided in
\gls{MOOSE}.